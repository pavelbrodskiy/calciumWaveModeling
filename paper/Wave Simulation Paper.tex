% Generated by GrindEQ Word-to-LaTeX 2015 
% LaTeX/AMS-LaTeX

\documentclass{article}

%%% remove comment delimiter ('%') and specify encoding parameter if required,
%%% see TeX documentation for additional info (cp1252-Western,cp1251-Cyrillic)
%\usepackage[cp1252]{inputenc}

%%% remove comment delimiter ('%') and select language if required
%\usepackage[english,spanish]{babel}

\usepackage{amssymb}
\usepackage{amsmath}

%%% remove comment delimiter ('%') and select graphics package
%%% for DVI output:
\usepackage[dvips]{graphicx}
%%% or for PDF output:
%\usepackage[pdftex]{graphicx}
%%% or for old LaTeX compilers:
%\usepackage[dvips]{graphics}

\begin{document}

%%% remove comment delimiter ('%') and select language if required
%\selectlanguage{spanish} 

\noindent 

\noindent 

\noindent 

\noindent 

\noindent TO DO

\begin{enumerate}
\item  Either experimental data needs to be included which strongly shows a difference between A and P compartment waves, or we need to find a paper to cite
\end{enumerate}

\noindent 

\noindent 

\noindent 

\noindent \eject 

\noindent 
\section{Title page}

\noindent Title: 

\noindent Authors: PA Brodskiy, C Narciso, Q Wu, A Jilkine, JJ Zartman

\noindent 
\section{Abstract }

\noindent 
\section{Introduction}

\begin{enumerate}
\item \begin{enumerate}
\item \textbf{ }Calcium signaling[1]

\item  Wing disc

\item  Calcium modeling[2]--[4]
\end{enumerate}
\end{enumerate}

\noindent 
\section{Materials and Methods}

\noindent 

\begin{tabular}{|p{0.2in}|p{3.9in}|p{0.3in}|} \hline 
 & \includegraphics*[width=2.78in, height=0.30in, keepaspectratio=false]{image1} & (1) \\ \hline 
\end{tabular}

where $J_{flux}$ represents the flux of ${Ca}^{2+}$ out of the ER through IP3R channels, $J_{SERCA}$ represents the flux of ${Ca}^{2+}$ into the ER through SERCA pumps, $J_{media}$ represents the flux of Ca${}^{2+}$${}^{ }$leaking in and out of the medium in response to cytoplasmic ${Ca}^{2+}$ concentration, and $D_{{Ca}^{2+}}$ represents the effective diffusivity of ${Ca}^{2+}$.

\begin{tabular}{|p{0.2in}|p{3.9in}|p{0.3in}|} \hline 
 & \includegraphics*[width=2.15in, height=0.36in, keepaspectratio=false]{image2} & (2) \\ \hline 
\end{tabular}



\begin{tabular}{|p{0.2in}|p{3.9in}|p{0.3in}|} \hline 
 & \includegraphics*[width=1.99in, height=0.36in, keepaspectratio=false]{image3} & (3) \\ \hline 
\end{tabular}



\begin{tabular}{|p{0.2in}|p{3.9in}|p{0.3in}|} \hline 
 & \includegraphics*[width=2.30in, height=0.30in, keepaspectratio=false]{image4} & (4) \\ \hline 
\end{tabular}



\begin{tabular}{|p{0.2in}|p{3.9in}|p{0.3in}|} \hline 
 & \includegraphics*[width=3.08in, height=0.37in, keepaspectratio=false]{image5} & (5) \\ \hline 
\end{tabular}



\begin{tabular}{|p{0.2in}|p{3.9in}|p{0.3in}|} \hline 
 & \includegraphics*[width=2.78in, height=0.30in, keepaspectratio=false]{image6} & (6) \\ \hline 
\end{tabular}



\begin{enumerate}
\item \begin{enumerate}
\item  Experimental stuff

\item  Derivation and repurposing of model

\begin{enumerate}
\item  Explain why we used sneyd form of SERCA equation and hofer form of Jflux equation---ideally we would fit to Cl8 cell data

\item  Explain why gamma distribution of signal makes sense in the context
\end{enumerate}

\item  Tuning of parameters

\begin{enumerate}
\item  Explain reasoning for each parameter and realistic biological range
\end{enumerate}

\item  Solving PDEs

\item  Possible numerical analysis
\end{enumerate}
\end{enumerate}

\noindent 
\section{Results}

\begin{enumerate}
\item \begin{enumerate}
\item \textbf{ }Variance of stochastic noise term directly results in a transition from no waves, to traveling waves, to random pulses

\item  Comparison of different phenotypes pulled from parameter sweep and analysis of biological implications

\begin{enumerate}
\item  Patterning of refractory period

\item  Patterning of steady-state concentration

\item  Biologically impossible results (ie. Loss of wave activity)
\end{enumerate}

\item  

\item  Formal numerical stability analysis should be conducted
\end{enumerate}
\end{enumerate}

\noindent 
\section{Discussion }

\noindent 
\section{Conclusion}

\noindent 
\section{Acknowledgments }

\noindent 
\section{References}

\noindent 
\section{[1] C. Narciso, Q. Wu, P. Brodskiy, G. Garston, R. Baker, A. Fletcher, and J. Zartman, ``Patterning of wound-induced intercellular Ca 2+ flashes in a developing epithelium,'' Phys. Biol., vol. 12, no. 5, p. 056005, 2015.[2] J. Keener and J. Sneyd, Mathematical Physiology: I: Cellular Physiology, 2nd edition. New York, NY: Springer, 2008.[3] J. Sneyd, B. T. Wetton, A. C. Charles, and M. J. Sanderson, ``Intercellular calcium waves mediated by diffusion of inositol trisphosphate: a two-dimensional model,'' Am. J. Physiol. - Cell Physiol., vol. 268, no. 6, pp. C1537--C1545, Jun. 1995.[4] T. H�fer, L. Venance, and C. Giaume, ``Control and Plasticity of Intercellular Calcium Waves in Astrocytes: A Modeling Approach,'' J. Neurosci., vol. 22, no. 12, pp. 4850--4859, Jun. 2002.Supporting Material}

\noindent 


\end{document}

